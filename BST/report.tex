\documentclass[UTF8]{ctexart}
\usepackage{geometry, CJKutf8}
\geometry{margin=1.5cm, vmargin={0pt,1cm}}
\setlength{\topmargin}{-1cm}
\setlength{\paperheight}{29.7cm}
\setlength{\textheight}{25.3cm}

% useful packages.
\usepackage{amsfonts}
\usepackage{amsmath}
\usepackage{amssymb}
\usepackage{amsthm}
\usepackage{enumerate}
\usepackage{graphicx}
\usepackage{multicol}
\usepackage{fancyhdr}
\usepackage{layout}
\usepackage{listings}
\usepackage{float, caption}

\lstset{
    basicstyle=\ttfamily, basewidth=0.5em
}

% some common command
\newcommand{\dif}{\mathrm{d}}
\newcommand{\avg}[1]{\left\langle #1 \right\rangle}
\newcommand{\difFrac}[2]{\frac{\dif #1}{\dif #2}}
\newcommand{\pdfFrac}[2]{\frac{\partial #1}{\partial #2}}
\newcommand{\OFL}{\mathrm{OFL}}
\newcommand{\UFL}{\mathrm{UFL}}
\newcommand{\fl}{\mathrm{fl}}
\newcommand{\op}{\odot}
\newcommand{\Eabs}{E_{\mathrm{abs}}}
\newcommand{\Erel}{E_{\mathrm{rel}}}

\begin{document}

\pagestyle{fancy}
\fancyhead{}
\lhead{周钰, 3230105681}
\chead{数据结构与算法第四次作业}
\rhead{Oct.20th, 2024}

\section{修改remove的思路函数}
首先,课堂上分析时指出原先的remove函数递归有重复操作,效率较低;

结合作业中的提示,想要修改后的remove函数对删除的节点有两个子节点的情况进行优化,利用BinaryNode* detachMin(BinaryNode* \&t)函数查找以 t 为根的子树中的最小节点,返回这个节点,并从原子树中删除这个节点,并且当要删除的节点具有两个子树时,通过这个函数返回的右子树最小节点将代替被删除节点;在函数BinaryNode* detachMin(BinaryNode* \&t)具体的实现过程中,采用递归;

对于remove函数的修改主要是在if语句中,当发现删除的元素有两个子节点时,将原先的仅替换值而不替换节点的部分修改为调用BinaryNode* detachMin(BinaryNode* \&t)函数进行节点替换;




\section{main函数的设计思路与测试结果}

\begin{enumerate}

    \item 我首先创建了一个二叉搜索树,先后向其中插入元素10,5,15,3,7,12,18;
    \item 第一步:先测试打印和二叉树基本的排序功能,打印二叉树全部元素;
    \item 第二步:测试查找最小和最大元素功能,并打印输出;
    \item 第三步:测试contains功能,先后查找元素7和20,并且找到输出Yes,反之输出No;  
    \item 第四步:测试删除功能,先后删除7和10,并且分别打印结果; 
    \item 第五步:测试清空树并且检查是否为空,打印结果;  
    \item 第六步:测试拷贝构造函数和赋值运算符,创建一个新的二叉搜索树bst2,向其中插入1,3,2,然后再创建bst3,bst4分别测试拷贝构造函数和赋值运算符,并分别打印结果;
    \item 第七步:测试移动构造函数和移动赋值运算符,创建新的二叉搜索树bst5和bst6用来分别测试移动构造函数和赋值运算符,并分别打印结果。

\end{enumerate}

\section{内存泄露检查}
没有测试异常情况时,我用 valgrind 进行测试,发现没有发生内存泄露。

\end{document}

%%% Local Variables: 
%%% mode: latex
%%% TeX-master: t
%%% End: 
