\documentclass[UTF8]{ctexart}
\usepackage{geometry, CJKutf8}
\geometry{margin=1.5cm, vmargin={0pt,1cm}}
\setlength{\topmargin}{-1cm}
\setlength{\paperheight}{29.7cm}
\setlength{\textheight}{25.3cm}

% useful packages.
\usepackage{amsfonts}
\usepackage{amsmath}
\usepackage{amssymb}
\usepackage{amsthm}
\usepackage{enumerate}
\usepackage{graphicx}
\usepackage{multicol}
\usepackage{fancyhdr}
\usepackage{layout}
\usepackage{listings}
\usepackage{float, caption}

\lstset{
    basicstyle=\ttfamily, basewidth=0.5em
}

% some common command
\newcommand{\dif}{\mathrm{d}}
\newcommand{\avg}[1]{\left\langle #1 \right\rangle}
\newcommand{\difFrac}[2]{\frac{\dif #1}{\dif #2}}
\newcommand{\pdfFrac}[2]{\frac{\partial #1}{\partial #2}}
\newcommand{\OFL}{\mathrm{OFL}}
\newcommand{\UFL}{\mathrm{UFL}}
\newcommand{\fl}{\mathrm{fl}}
\newcommand{\op}{\odot}
\newcommand{\Eabs}{E_{\mathrm{abs}}}
\newcommand{\Erel}{E_{\mathrm{rel}}}

\begin{document}

\pagestyle{fancy}
\fancyhead{}
\lhead{周钰, 3230105681}
\chead{四则运算器}
\rhead{2024.12.22}

\section{设计思路}
处理中缀运算式,将符号和数字分开来处理,建立了op和nums两个stack;

处理op时,分为+,-,*,/四则运算符号,和(,)左右括号,和其他不合法字符。按照中缀运算式的思路,先检查是否含有不合法字符并报错,随后对四则运算符号设立了优先级,再对含有左右括号的情况进行讨论。

遇到左括号时,将其直接塞入stack中;遇到)时,弹出op stack里顶部运算符直至遇到(为止;

处理nums时,先将运算式中的数字提取出来,然后字符串转换成数字,最后压入nums stack;

在计算时,运用定义的优先级和op stack,nums stack设计calculate函数,返回double类型的值。

\section{异常情况处理果}

异常情况有很多,根据作业要求和实际设计时遇到的问题,可以分为:

1.计算式中除数为0。在calculate中进行检验。


2.连续使用运算符/开头出现运算符号/出现其他符号。在得到输入中缀表达式后,进行合法性检验。


3.左右括号数量不匹配。在处理()时,对他们分别计数,数量不一致时报错。


\section{考虑负数情况}
对于负数情况,我将负数x变为‘0-x’处理。

由于设置了运算优先级和对连续使用运算符的异常情况判断,计算负数时必须添加括号,例如 1+(-2),(-2)/1 等情况。

\end{document}

%%% Local Variables: 
%%% mode: latex
%%% TeX-master: t
%%% End: 
