\documentclass[UTF8]{ctexart}
\usepackage{geometry, CJKutf8}
\geometry{margin=1.5cm, vmargin={0pt,1cm}}
\setlength{\topmargin}{-1cm}
\setlength{\paperheight}{29.7cm}
\setlength{\textheight}{25.3cm}

% useful packages.
\usepackage{amsfonts}
\usepackage{amsmath}
\usepackage{amssymb}
\usepackage{amsthm}
\usepackage{enumerate}
\usepackage{graphicx}
\usepackage{multicol}
\usepackage{fancyhdr}
\usepackage{layout}
\usepackage{listings}
\usepackage{float, caption}

\lstset{
    basicstyle=\ttfamily, basewidth=0.5em
}

% some common command
\newcommand{\dif}{\mathrm{d}}
\newcommand{\avg}[1]{\left\langle #1 \right\rangle}
\newcommand{\difFrac}[2]{\frac{\dif #1}{\dif #2}}
\newcommand{\pdfFrac}[2]{\frac{\partial #1}{\partial #2}}
\newcommand{\OFL}{\mathrm{OFL}}
\newcommand{\UFL}{\mathrm{UFL}}
\newcommand{\fl}{\mathrm{fl}}
\newcommand{\op}{\odot}
\newcommand{\Eabs}{E_{\mathrm{abs}}}
\newcommand{\Erel}{E_{\mathrm{rel}}}

\begin{document}

\pagestyle{fancy}
\fancyhead{}
\lhead{周钰, 3230105681}
\chead{数据结构与算法第六次作业}
\rhead{Oct.20th, 2024}

\section{AVLtree的设计思路}
因为AVLtree需要保证左右子树的高度差不超过1,所以先对insert函数进行修改,在原有的插入操作后,进行平衡调整,使得每一次插入后都满足平衡条件,并且更新高度差;

在此之后,对remove函数进行修改:因为第五次作业没有实现节点替换,所以先把节点替换这一功能完成;之后,在原有的remove函数上进行平衡调整和更新高度差。

\section{平衡调整}


   平衡调整分为左旋,右旋和双旋,并且旋转过后需要更新高度差;
   
   对于左旋,如果一个节点的height为2,并且右孩子height为1,进行左旋,也就是改变当前节点和父节点的指针关系;右旋同理;
   
   当插入一个元素对于其父子树和父子树的父子树都破坏了平衡原则,那么需要进行双旋,具体就是先进行左旋,再进行右旋或者先右旋再左旋。   
   

\end{document}

%%% Local Variables: 
%%% mode: latex
%%% TeX-master: t
%%% End: 
