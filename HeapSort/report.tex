\documentclass[UTF8]{ctexart}
\usepackage{geometry, CJKutf8}
\geometry{margin=1.5cm, vmargin={0pt,1cm}}
\setlength{\topmargin}{-1cm}
\setlength{\paperheight}{29.7cm}
\setlength{\textheight}{25.3cm}

% useful packages.
\usepackage{amsfonts}
\usepackage{amsmath}
\usepackage{amssymb}
\usepackage{amsthm}
\usepackage{enumerate}
\usepackage{graphicx}
\usepackage{multicol}
\usepackage{fancyhdr}
\usepackage{layout}
\usepackage{listings}
\usepackage{float, caption}

\lstset{
    basicstyle=\ttfamily, basewidth=0.5em
}

% some common command
\newcommand{\dif}{\mathrm{d}}
\newcommand{\avg}[1]{\left\langle #1 \right\rangle}
\newcommand{\difFrac}[2]{\frac{\dif #1}{\dif #2}}
\newcommand{\pdfFrac}[2]{\frac{\partial #1}{\partial #2}}
\newcommand{\OFL}{\mathrm{OFL}}
\newcommand{\UFL}{\mathrm{UFL}}
\newcommand{\fl}{\mathrm{fl}}
\newcommand{\op}{\odot}
\newcommand{\Eabs}{E_{\mathrm{abs}}}
\newcommand{\Erel}{E_{\mathrm{rel}}}

\begin{document}

\pagestyle{fancy}
\fancyhead{}
\lhead{周钰, 3230105681}
\chead{数据结构与算法第七次作业}

\section{设计思路}
1.heapsort 函数
其功能是对传入的 std::vector<typename> 进行堆排序;
实现细节:
首先,通过 percDown 函数从最后一个非叶子节点开始向上构建最大堆。
然后,将堆顶元素(最大值)与最后一个元素交换,减小堆的大小,并重新调整堆以保持最大堆的性质。
重复上述过程,直到堆的大小为1,即完成排序。

2.percDown 函数
其功能是用于构建最大堆和删除最大元素后调整堆结构。
实现细节:
从指定的节点开始,将其与子节点中较大的值进行比较,如果当前节点值小于子节点中的最大值,则交换它们。
继续这个过程,直到找到正确的位置或者到达叶子节点。

\section{测试流程}

1.生成测试数据:分别生成长度为1,000,000的随机序列、有序序列、逆序序列和部分元素重复序列。其中,使用随机数生成器和特定的分布来生成随机序列,使用简单的循环生成有序序列和逆序序列,通过限制随机数生成器的取值范围来生成部分元素重复序列。


2.执行排序:对每种类型的序列,分别使用自定义的 heapsort 函数和 STL 的 std::sort\_heap 函数进行排序,并使用chrono记录并比较两种排序方法的执行时间。


\section{效率对比}
\begin{table}[h!]
	\begin{center}
		\begin{tabular}{|l|c|r|} 
			\hline
			\textbf{ } & \textbf{my heapsort time} & \textbf{std::sort\_heap time}\\
			\hline
			random sequence & 0.359594s & 0.965745s\\
			\hline
			ordered sequence & 0.243334s & 0.83748s\\
			\hline
			reverse sequence & 0.210941s & 0.74334s\\
			\hline
			repetitive sequence & 0.252258s & 0.804494s\\
			\hline
		\end{tabular}
	\end{center}
\end{table}
\section{原因分析}
heapsort()的排序时间在四种情况下都小于std::sort\_heap的排序时间。

分析如下:

 
1.heapsort()提供了更多的控制,因为它包含了堆的构建和排序的完整过程,用户可以自定义堆的行为(例如,通过修改 percDown 函数来实现最小堆排序);


std::sort\_heap():灵活性较低,因为它只负责排序已经构建好的堆,用户需要自己管理堆的构建和可能的其他操作(如插入和删除)。


2.heapsort()适用于需要直接对一个数组或 std::vector 进行堆排序的场景。


  而std::sort\_heap():用户需要先使用 std::make\_heap() 或其他方式构建堆,然后调用 std::sort\_heap()。
  
3.heapsort()作为普通函数被优化,

而std::sort\_heap() 作为一个模板函数,可能无法像普通函数那样被优化。

\end{document}

%%% Local Variables: 
%%% mode: latex
%%% TeX-master: t
%%% End: 
