\documentclass[UTF8]{ctexart}
\usepackage{geometry, CJKutf8}
\geometry{margin=1.5cm, vmargin={0pt,1cm}}
\setlength{\topmargin}{-1cm}
\setlength{\paperheight}{29.7cm}
\setlength{\textheight}{25.3cm}

% useful packages.
\usepackage{amsfonts}
\usepackage{amsmath}
\usepackage{amssymb}
\usepackage{amsthm}
\usepackage{enumerate}
\usepackage{graphicx}
\usepackage{multicol}
\usepackage{fancyhdr}
\usepackage{layout}
\usepackage{listings}
\usepackage{float, caption}

\lstset{
    basicstyle=\ttfamily, basewidth=0.5em
}

% some common command
\newcommand{\dif}{\mathrm{d}}
\newcommand{\avg}[1]{\left\langle #1 \right\rangle}
\newcommand{\difFrac}[2]{\frac{\dif #1}{\dif #2}}
\newcommand{\pdfFrac}[2]{\frac{\partial #1}{\partial #2}}
\newcommand{\OFL}{\mathrm{OFL}}
\newcommand{\UFL}{\mathrm{UFL}}
\newcommand{\fl}{\mathrm{fl}}
\newcommand{\op}{\odot}
\newcommand{\Eabs}{E_{\mathrm{abs}}}
\newcommand{\Erel}{E_{\mathrm{rel}}}

\begin{document}

\pagestyle{fancy}
\fancyhead{}
\lhead{周钰, 3230105681}
\chead{数据结构与算法第四次作业}
\rhead{Oct.20th, 2024}

\section{测试程序的设计思路}
我首先创建了一个链表,含有数字4,5;  

第一步:先测试右值插入。向尾部插入6,再插入7;向头部插入3,再插入2。采用打印的方式检查结果是否正确。  

第二步:检查列表的大小和是否为空链表。  

第三步:测试++功能和--功能。分别测试前置自增运算符、后置自增运算符、前置自减运算符、后置自减运算符。  

第四步:测试访问首尾元素,通过输出验证正确性。  

第五步:测试删除链表的首尾元素,并使用迭代器遍历链表,输出结果验证删除操作是否正确。  

第六步:使用clear方法清空链表,并输出链表大小,验证清空操作是否正确。  

第七步:测试拷贝构造。先插入3和4,再创建一个链表copyList,同时把myList赋值给它。使用迭代器遍历copList,输出结果验证拷贝构造是否正确。  

第八步:测试赋值运算符。创建anotherList,再将myList赋值给anotherList。使用迭代器遍历anotherList,输出结果以验证赋值运算符是否正确。 

\section{测试的结果}

测试结果一切正常。

我用 valgrind 进行测试,发现没有发生内存泄露。

\section{(可选)bug报告}

我发现了一个 bug,触发条件如下:

\begin{enumerate}
    \item 首先……
    \item 然后……
    \item 此时发现……
\end{enumerate}

据我分析,它出现的原因是:
cai'yong'da'gun
\end{document}

%%% Local Variables: 
%%% mode: latex
%%% TeX-master: t
%%% End: 
